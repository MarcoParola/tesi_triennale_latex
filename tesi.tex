\documentclass[a4paper]{article}

\usepackage[T1]{fontenc}
\usepackage[utf8]{inputenc}
\usepackage[italian]{babel}
\usepackage{frontespizio}
\usepackage{graphicx}

\begin{document}
	
	
\begin{frontespizio} 
 \Preambolo{\renewcommand{\frontpretitlefont}{\fontsize{14}{12}\scshape}}
\Universita{Pisa} 
\Facolta{Ingegneria} 
\Corso[Laurea]{Ingegneria Informatica} 
\Annoaccademico{2019--2020} 
\Titolo{Titolo} 
\Sottotitolo{Sottotitolo}
\Filigrana [height=4cm,before=0.28,after=1]{./images/stemma_unipi.png} 


\Rientro{1cm} 
\Candidato{540779 Marco Parola} 
\Relatore{Dr. Francesco Pistolesi}
\Punteggiatura{} 
 
\end{frontespizio}

	\tableofcontents

	\clearpage

	\section{Introduzione}
Lo scopo di questa tesi è analizzare i movimenti compiuti durante il sollevamento di un carico elevato mediante dispositivi indossabili che contengono al loro interno sensori. \\
L’utilità di questo tipo di analisi diventa evidente quando andiamo a considerare i dati provenienti dagli enti di previdenza sociale relativi al tipo e ai numeri degli infortuni sul lavoro e delle malattie professionali. Questi dati evidenziano infatti come, a livello mondiale, a prevalere siano malattie professionali classificabili come disturbi Muscolo-Scheletrici.
La mansione del sollevamento di un carico aumenta in maniera considerevole l'indice di rischio di questo tipo di disturbi.
Considerando poi che quasi un terzo dei lavoratori dichiara di svolgere quotidianamente questo tipo di operazione, è lecito pensare che avendo uno strumento capace di analizzare il modo in cui si effettua questo task potrebbe permettere di comprendere il problema più a fondo e cercare una soluzione mirata.

	\clearpage

	\section{Raccolta dati}
Per poter eseguire uno studio del rischio che corre una persona, sollevando un carico elevato, si affronta una prima fase in cui si raccolgono i dati relativi alla movimentazione di quest'ultima. \\
Dunque si rende necessario uno strumento che permetta di raccogliere e catalogare informazioni con cui poter ricostruire il movimento ed in una seconda fase eseguire un analisi per poter quantificare il rischio.

	\subsection{Strumento per la raccolta ed il salvataggio di dati}

	\subsubsection{Caratteristiche}
Definiamo una linea da seguire durante lo storage delle informazioni attraverso lo strumento che definiremo: miniinvasività! \\
Questa scelta è stata presa per...
 
	\subsubsection{Applicazione Android per la rccolta}
La soluzione che abbiamo trovato che rispecchi 

\end{document}